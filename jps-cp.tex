\documentclass[12pt,a4paper]{article}
\usepackage[margin=1in]{geometry}

\usepackage{lineno,hyperref}
\usepackage{comment}
\usepackage{upgreek}
\usepackage{graphics}
\usepackage[subrefformat=parens]{subcaption}
\usepackage{siunitx}
\usepackage{txfonts} %Please comment out this line unless the txfonts package is availabe in your LaTeX system.
\usepackage{authblk}
\usepackage{footmisc} 
%\usepackage{amsmath}
\usepackage{amssymb}
\usepackage{enumitem}

\usepackage{background}
\usepackage{listings}
\definecolor{mygreen}{rgb}{0,0.6,0}
\definecolor{mygray}{rgb}{0.5,0.5,0.5}
\definecolor{mymauve}{rgb}{0.58,0,0.82}
\lstset{ %
  backgroundcolor=\color{white},   % choose the background color; you must add \usepackage{color} or \usepackage{xcolor}
  basicstyle=\ttfamily\footnotesize, % the size of the fonts that are used for the code
  breakatwhitespace=false,         % sets if automatic breaks should only happen at whitespace
  breaklines=false,                % sets automatic line breaking
  captionpos=b,                    % sets the caption-position to bottom
  commentstyle=\color{mygreen},    % comment style
  deletekeywords={...},            % if you want to delete keywords from the given language
  escapeinside={\%*}{*)},          % if you want to add LaTeX within your code
  extendedchars=true,              % lets you use non-ASCII characters; for 8-bits encodings only, does not work with UTF-8
  frame=single,                    % adds a frame around the code
  keepspaces=true,                 % keeps spaces in text, useful for keeping indentation of code (possibly needs columns=flexible)
%  keywordstyle=\color{blue},       % keyword style
  language=Octave,                 % the language of the code
  morekeywords={*,...},            % if you want to add more keywords to the set
  numbers=left,                    % where to put the line-numbers; possible values are (none, left, right)
  numbersep=5pt,                   % how far the line-numbers are from the code
  numberstyle=\tiny\color{mygray}, % the style that is used for the line-numbers
  rulecolor=\color{black},         % if not set, the frame-color may be changed on line-breaks within not-black text (e.g. comments (green here))
  showspaces=false,                % show spaces everywhere adding particular underscores; it overrides 'showstringspaces'
  showstringspaces=false,          % underline spaces within strings only
  showtabs=false,                  % show tabs within strings adding particular underscores
  stepnumber=1,                    % the step between two line-numbers. If it's 1, each line will be numbered
  stringstyle=\color{mymauve},     % string literal style
  tabsize=2,                       % sets default tabsize to 2 spaces
  title=\lstname                   % show the filename of files included with \lstinputlisting; also try caption instead of title
}
%\usepackage[whole]{bxcjkjatype} % Just used to enable Japanese characters for temporary use
%\usepackage{xcolor} % For text color

\sisetup{
  range-phrase=~--~,
  range-units=single
}

%\modulolinenumbers[5]

\newcommand{\figref}[1]{Fig.~\ref{#1}}
\newcommand{\Figref}[1]{Figure~\ref{#1}}
\newcommand{\tabref}[1]{Tab.~\ref{#1}}
\newcommand{\secref}[1]{Sec.~\ref{#1}}
\newcommand{\Eqref}[1]{Eq.~\eqref{#1}}
\newcommand{\citeref}[1]{Ref.~\cite{#1}}

\bibliographystyle{elsarticle-num}
%
%
%\inst{$^{1}$Institute of Particle and Nuclear Studies, KEK, Tsukuba, Ibaraki 305-0801, Japan \\
%$^{2}$J-PARC Center, Tokai, Ibaraki 319-1195, Japan\\
%$^{3}$Institute of High Energy Physics, Chinese Academy of Sciences, 100049, Beijing, China\\}
%
%\email{yusuke.uchiyama@kek.jp}
%
%\recdate{\today}
%

%
%\kword{radiation shield, high-power targetry, muon source}

\begin{document}

\SetBgContents{COMET Internal Note}      % a watermark
\SetBgPosition{current page.center}
\SetBgAngle{0}                                    % rotate
\SetBgColor{gray}                                 % color
\SetBgScale{2.4}                                  % scale
\SetBgHshift{1.6cm}                                   % location x=0 for center
\SetBgVshift{5.6cm} 

\title{Technical details of CS inner shield system}
%
\author{Yusuke \textsc{Uchiyama}$^{1,2}$, Yoshinori \textsc{Fukao}$^{1,2}$, Shunsuke \textsc{Makimura}$^{1,2}$, Satoshi~\textsc{Mihara}$^{1,2}$, \\
\vspace{-2mm}
Shaojing \textsc{Hou}$^{3}$, Xiaoyan \textsc{Ma}$^{3}$, Ye \textsc{Yuan}$^{3}$, Yao \textsc{Zhang}$^{3}$\\
\vspace{5mm}
{\footnotesize
$^{1}$Institute of Particle and Nuclear Studies, KEK, Tsukuba, Ibaraki 305-0801, Japan \\
$^{2}$J-PARC Center, Tokai, Ibaraki 319-1195, Japan\\
$^{3}$Institute of High Energy Physics, Chinese Academy of Sciences, 100049, Beijing, China\\}
}

\maketitle
\begin{abstract}
To provide a high-intensity muon beam to the COMET experiment,  pions produced from the collision of \SI{8}{\GeV} protons from the J-PARC Main Ring with a graphite target are collected using a superconducting pion capture solenoid (CS). It is necessary to install a massive radiation shield inside the CS to protect the coils from the quench as well as radiation damage.
This note describes the details of the CS inner shield system.
%%The CS inner shield system is placed in a high and graded magnetic field and a vacuum volume of $\mathcal{O}(10^{-5}~\mathrm{Pa})$. 
The main shield materials are decided to be stainless steel and copper with a thickness of \SI{420}{\mm} with a total weight of \SI{13}{\tonne}. 
%%The total heat load on the shield with a \SI{3.2}{\kW} beam is expected to be \SI{1}{\kW}, which is removed by air flowing through the bulk of the shield.  It also serves as the interface for an exchangeable target system with a precise position reference. 
The shield is designed as a part of a complex vacuum system and
together with peripheral equipment, it forms a high-power targetry and beam interception system.
The whole system is designed to be exchangeable with semi-remote handling after Phase-I of COMET, at which the components will be highly radio-activated.
\end{abstract}

\tableofcontents

\section{Introduction}
%The COMET experiment \cite{comet} in the J-PARC Hadron Experimental Facility searches for the lepton-flavor-violating muon-to-electron conversion process using \SI{8}{\GeV} protons from the Main Ring. 
In COMET Phase-I, we aim for the single-event sensitivity of \num{3e-15} for the lepton-flavor-violating muon-to-electron conversion process and need \SI{>1e9}{\per\second} negative muons to be stopped in aluminum targets.
To make such a high-intensity negative muon beam, we use a \SI{490}{\mm} long graphite target hit by the \SI{8}{\GeV} primary proton beam of \SI{3.2}{\kW}.
Pions from the production target are collected and extracted to the experiment in the backward direction by a high and graded magnetic field formed by a superconducting pion capture solenoid (CS) \cite{pcs}.
Figure~\ref{fig:overview} shows the schematic view of the pion production system in the COMET beam room.

Secondary particles from the collision of the proton beam with the production target induce heat in the cold mass of the CS and subsequently a quench of the superconductor. Exposure to irradiation also leads to degradation of thermal conductivity in the conduction-cooled superconducting coils, which limits the continuous operation of the CS system \cite{inner-shield}. Therefore, it is necessary to install a massive radiation shield inside the warm bore of CS.
In this note, we describe the technical and engineering design of the CS inner shield system. 

\begin{figure}[tb]
\includegraphics[width=\textwidth]{fig/Overview.pdf}
\caption{Schematics of the COMET beam room and the pion production system. It is located in the basement of the Hadron South Experimental Building in J-PARC. Trajectories in red, green, and light blue are protons, negative pions, and negative muons, respectively. One hundred \SI{8}{\GeV} protons are injected into the production target in this figure. The other particles are omitted.}
\label{fig:overview}
\end{figure}

\section{Requirements for the CS inner shield}
The CS inner shield system is placed in a harsh environment of photon and neutron fluxes of $F_\gamma \sim \mathcal{O}(10^9)~\unit{\per\square\cm\per\second}$ and $F_n \sim \mathcal{O}(10^{10})~\unit{\per\square\cm\per\second}$, respectively, in a high and graded magnetic field ranging \qtyrange{4.9}{2.5}{\tesla}, and in a vacuum volume of $\mathcal{O}(10^{-4})$~\unit{\Pa}. 
The requirements for the CS inner shield system are the following:
\begin{enumerate}
    \item it should suppress the heat load to the CS superconducting coils by the secondary particle irradiation below \SI{40}{\watt};
    \item it should suppress the radiation damage to the coils down to the neutron fluence $\Phi_n < 10^{17}~\unit{\per\square\cm}$;
    \item it should keep the temperature of its outer faces below \SI{55}{\degreeCelsius};
    \item it should seal the vacuum of the warm bore of CS and provide an interface for the production target system to be placed with a precision of \SI{0.5}{\mm};
    \item it should be exchangeable even after highly radio-activated ($\mathcal{O}(10)$~\unit{\milli\sievert\square\meter\per\hour});
    \item and it should not interfere in the pion transportation in the backward direction.
\end{enumerate}
Requirement (1) comes from the cooling capacity of the refrigerator for the entire magnet system of the COMET Phase-I (\SI{110}{\watt}) and it can be extended by upgrading the refrigerator for Phase-II.
Requirement (2) comes from the limit of degradation of the thermal conductivity in the aluminum thermal path in the coil. Note that this degradation can be recovered by a thermal cycle to room temperature \cite{inner-shield}. Therefore, the neutron damage limits the continuous operation of the CS system but not the total lifetime.



\section{Design of shield system}
The design and specifications for the CS inner shield system are described in this section.
More detailed drawings are given in Appendix \ref{sec:drawings}.

Please note that unless specified ``upstream/downstream'' indicate those with respect to the primary proton beam with regard to the CS inner shield system; namely the beam dump side is downstream. It could be opposite for other systems (such as secondary beam line and detectors).


\subsection{Overview of the shield design}
In the early stage of the design of the experiment as in Refs.~\cite{inner-shield,comet}, a tungsten-based alloy was considered as the main material of the shield.
This is because the design at that time was based on the maximum beam power of \SI{56}{\kW} in Phase-II.
However, it is technically challenging to construct the entire CS inner shield with tungsten.
On the other hand, there are more choices of materials for the \SI{3.2}{\kW} beam of Phase-I.
We decided to make it with lighter and low-cost material while keeping the possibility of upgrading it towards Phase-II. 
%Therefore, it must be designed to be exchangeable after the Phase-I.


\begin{figure}[tb]
\includegraphics[width=\textwidth]{fig/CSInnerShieldDesign1.pdf}
\caption{The design of the CS inner shield. The shield is segmented into 12 modules in the azimuth direction with modules made of SS304 and OFC. The total weight is about \SI{13}{\tonne}.}
\label{f1}
\end{figure}

Figure~\ref{f1} shows the updated design of the CS inner shield.
Considering the shielding power, material properties, availability, and cost, 
 stainless steel (SS304) and oxygen-free copper (OFC) are chosen as the main shield material.
While the outer diameter is defined to be \SI{1140}{\mm} by the radius of the warm bore of the CS (\SI{1200}{\mm}), the inner diameter is determined to be \SI{300}{\mm} from the trajectories of pions to be collected in the central \SI{4.9}{\tesla} field. 
The thickness of 420 mm corresponds to \num{2.7} nuclear interaction length for copper. 
The barrel part of the shield is segmented into 12 modules in the azimuth direction. Each module is 
manufactured individually and then they are assembled with upstream and downstream end-plates.

While the CS inner shield is concentric with the CS, the production target is not
because the proton beam is incident to the CS with an angle and bent inside the CS by its field. 
Therefore, the distribution of secondary particles is asymmetric in the azimuth direction; i.e., necessary shielding power is different for different modules.
The two segments exposed to the highest irradiation are occupied by the OFC modules, which have a higher density and shorter nuclear interaction and radiation lengths than the SS304 modules.
For weight balance as well as the balance of the axial force by eddy current applied to the OFC modules in case of a quench of CS, two more OFC modules are placed at mirror symmetric positions.

In the original design, the shape was identical for all modules with a tapered structure both in the upstream and downstream of the target to pass the bent beam.
 The tapered structure in an azimuthal symmetric way is necessary in the upstream side for the pion collection, but it is not necessary in the downstream side.
Via MARS simulations, we identified the weak point to shield radiation from the target to the outside the beam room, especially aboveground: the opening angle of the CS inner shield at the downstream side of target.
Five SS304 modules placed at the top side, where the beam is away, were modified in 2024 to have a straight shape at the downstream side
 as shown in Fig.~\ref{fig:a} to effectively block the radiation going upwards. %to reduce the air dose rate outside the beam room, especially above ground.
%We proposed to modify the taper shape to a straight one for the top five modules (see Fig.~\ref{fig:CSInnerShield_Design}) to effectively block the radiation going upwards.  

\begin{figure}[tbp]
\centering
\includegraphics[width=1\linewidth]{fig/CSInnerShield_Radiation.pdf}
\caption{Results of MARS calculation with the modified PCS inner shield design. (a) Prompt does at the first (ground) floor. The maximum does decreases from \SI{190}{\micro\sievert\per\hour} to \SI{120}{\micro\sievert\per\hour}. Note that this calculation was performed without placing the dump collimator shield. (b) Energy deposition to the PCS at different regions of the coils. (c) Ratio of the energy deposition with the modified design to that with the original design.}
\label{fig:CSInnerShield_Radiation}
\end{figure}
Figure~\ref{fig:CSInnerShield_Radiation} shows the results of radiation calculation with the modified design.  This modification leads to a 40\% reduction in the radiation level aboveground with a marginal 5\% increase in the damage to the coils.
We also performed a study using ICEDUST and confirmed a negligible impact on the detector backgrounds.
%Based on these results, the collaboration approved the proposal and we immediately asked the manufacturer the design change.
%Since the change is just a modification in a machining angle, which results in an increased yield rate of the already-procured raw material, they kindly accepted it for free.  


The total weight with this design is about \SI{13}{\tonne}.
The heat load on the shield is expected to be \SI{1}{\kW} in total and \SI{160}{\watt} in a module at maximum
at a beam power of \SI{3.2}{\kW}.  
The heat can be removed by a flow of room-temperature air through the shield body. 


\begin{figure}[tb]
\centering
\begin{minipage}[b]{0.45\columnwidth}
    \centering
    \includegraphics[width=1\columnwidth]{fig/CSInnerShieldDesign2.pdf}
    \caption{Cross section of the SS304 modules.}
    \label{fig:a}
\end{minipage}
\begin{minipage}[b]{0.54\columnwidth}
    \centering
    \includegraphics[width=1\columnwidth]{fig/CSInnerShieldDesign3.pdf}
    \caption{Detailed design of the OFC module.}
    \label{fig:b}
\end{minipage}
\end{figure}
%

\subsection{Design of SS304 module}
The SS304 module consists of SS304 shields and related equipment (thermal shield plates, blackbody plate, cooling pipes, cooling pipe covers, connection plates). The external dimensions are long block shapes (length 1655 mm) covering an azimuthal angle of 30 degrees from the beam axis, with a weight of approximately \SI{1}{\tonne} per module. There are two types of shield units with different shapes and details of the two units with the same shape are slightly different.

\subsubsection{Specification}
\paragraph{Material}
Forged material of non-magnetic SS304 with a density of \SI{7.6}{\gram\per\cubic\cm} or higher\footnote{The actual density of the material used in the 2024 production was \SI{7.84}{\gram\per\cubic\cm}.} shall be used. Demagnetization treatment such as solution heat treatment shall be performed. Mill sheets for the main materials shall be provided.

\paragraph{Surface treatment} 
Except for bolts and blackened areas,  electrochemical polishing and degreasing after machining (in addition if necessary buff polishing to Ra3.2 or less (\#400 equivalent) before electropolishing) shall be performed. The outgassing rate is supposed to be  \SI{3e-7}{\pascal \cubic\metre\per\square\metre\per\second}). 

\paragraph{Cooling holes}
Each SS304 shield requires air or water cooling\footnote{Air cooling is expected in the Phase-I operation, but it is designed so that water cooling is also possible. If it is necessary, for instance, in a higher beam power operation in future, we can switch to the water cooling from outside of the beam room.} around the outer cylinder. Flow paths (through-holes with a diameter 16 mm) in each SS304 shield shall be made with a gun drill. The air/water route shall be one round trip in the longitudinal direction. Weld a lid (cooling pipe covers) to connect the return path on the upstream side. Tolerance for the holes is 0.5 mm. If the holes are made by the two sides, match the hole at the center within 1 mm. 
SS316L seamless pipes (1/2 inch) at least 200 mm long on the downstream side shall be joined. The joined pipes will be extended later by automatic welding to the outside of the vacuum volume via feedthroughs. The whole cooling path must pass the following tightness tests:
\begin{enumerate}[itemsep=3pt, parsep=0pt]
\item PT test
\item Leak test: \SI{<1.0e-9}{\pascal\per\cubic\metre\per\second} with a vacuum hood method
\item Pressure test: apply \SI{1.5}{\mega\pascal} to the cooling piles using water and confirm no pressure drop within 1 hour.
\end{enumerate}

\paragraph{Temperature measurement}
The temperature of each SS shield will be measured with two E-type thermocouples. Taps and grooves for fixing and wiring the thermocouples shall be made. 

\paragraph{Blackbody plate}
 To absorb the radiant heat of the target, blackbody plates with blackbody treatment (equivalent to RAYDENT LSL-BL) shall be bolt-connected to the inner side. The surface roughness of the contact faces shall be Ra3.2 or better for good thermal contact. 

\paragraph{Thermal shield}
\SI{0.5}{\mm} thick SS304 thermal isolation shields shall be bolt-connected to the the outer side to reflect back the radiant heat of the main body. They shall be electrochemical polished to improve the albedo.    %Multiple pieces are acceptable as long as they cover the entire outer cylinder surface. 

\paragraph{Connection plates}
If the adjacent module is a copper one, fasten the adjacent surface with SS304 connection plates.

\paragraph{Bolt fastening}
Bolts with through holes shall be used if the tapped part is not through. Lifting taps shall be plugged with sealing plugs. Ensure no grease is applied to taps in contact with the vacuum or use radiation-tolerant high-vacuum grease, MORESCO RG-42R-1. Consider anti-loosening measures for all fastening bolts.

\subsection{Design of OFC module}
Details of OFC modules are different from the SS304 one. The OFC module shown in Fig.~\ref{fig:b} has a more complex design due to its weaker mechanical strength and higher electrical conductivity.  %Here, the detailed design of more complicated OFC module is described.

The main copper body consists of three parts. 
Because copper has a high electrical conductance, a large eddy current force is applied when the surrounding magnetic field suddenly changes, for instance, when the CS is quenched. The force can be suppressed by dividing the continuous copper body by a low-conductance material. With an FEM analysis, we decided to divide one module into two parts;
the left and right parts are separated by a \SI{10}{\mm} SS304 plate.
The outer part serves as a cooling plate and assembled with bolted connection to the left and right parts. 
%The cooling plate is coupled with a 1/2-inch stainless steel pipe by the hot isostatic pressing or vacuum brazing technique.
%If it is necessary, for instance, in a higher beam power operation in future, we can switch to the water cooling from outside of the beam room.
Sufficient thermal conductance with this method was demonstrated with a prototype.
%The temperature of each part of each module will be monitored with two E-type thermocouples during the beam operation.

%The inner side, facing the target, is made of another part. To absorb the radiant heat of the target, this part is processed with a black-body surface treatment and it is also bolt-connected to the main body.
%The outer side is covered with a \SI{0.5}{\mm} thick SS304 thermal isolation shield to reflect back the radiant heat of the main body.
%It is electro-polished to improve the albedo.   
%Not only the thermal shield but also all surfaces of other parts are electro-polished for low outgassing rate of about \SI{3e-7}{\Pa\cubic\m/\square\m/\second}.

\subsubsection{Specification}
\paragraph{Material}
Rolled material of oxygen-free copper (C10200) with a density of 8.9 g/cm³ or higher shall be used. Mill sheets for the main materials shall be provided.

\paragraph{Surface treatment} 
Except for bolts, blackened, and brazed areas,  buff polishing to Ra3.2 or less (\#400 equivalent)  shall be performed after machining, followed by acetone cleaning and then pickling. 

\paragraph{Cooling plate}
Each copper module requires air or water cooling around the outer cylinder. SS316L seamless pipes (1/2 inch) shall be joined to a cooling plate of copper by hot isostatic pressing (HIP) or vacuum brazing (silver/palladium) and the cooling plate shall be bolt-connected to the main copper shield L and R. The surface roughness of the contact faces shall be Ra3.2 or better for good thermal contact. Extend the other end of the joined pipes at least 200 mm from the copper shield for later extension by automatic welding. 
The whole cooling path must pass the same tightness tests as the SS304 module.


\paragraph{Temperature measurement}
The temperature of each copper part (left-, right-parts, and cooling plate) will be measured with two E-type thermocouples. Taps and grooves for fixing and wiring the thermocouples shall be made. 

\paragraph{Bolt fastening}
Helicoil inserts shall be used for all bolt fastening parts in copper material, adopting a tightening torque of 0.5 T series. Other specifications are the same as those for SS304 module.

\paragraph{Connection plates}
Each copper module shall be fastened to adjacent modules with SS304 connection plates.

\paragraph{Blackbody plate and thermal shield}
Refer to those for SS304 module.

%The SS304 module consists of a single main bulk body, the black-body plate, and the thermal shield.
%The path of cooling media is made directly in the outer region of the bulk with a gun drill. 

\subsection{System}
The CS inner shield is designed as a part of a complex vacuum system. Figure~\ref{fig:system} depicts the entire system.
Together with peripheral equipment, it forms a high-power targetry and beam interception system.

\begin{figure}[tb]
\includegraphics[width=\textwidth]{fig/CSInnerSystem.pdf}
\caption{The CS inner shield system.}
\label{fig:system}
\end{figure}




\subsubsection{Downstream flange ASSY}
Figure~\ref{fig:CSInnerShield_downstreamFlange} shows the design of the downstream flange ASSY.
\begin{figure}[tbp]
\centering
\includegraphics[width=1\linewidth]{fig/CSInnerShield_DownstreamFlange.pdf}
\caption{Schematics of the downstream flange ASSY.}
\label{fig:CSInnerShield_downstreamFlange}
\end{figure}
The shield is supported by cantilever beams from the downstream side and connected to vacuum flanges placed outside the CS. A mechanical flange supports the load at the downstream side and it provides the position reference with a precision of \SI{0.5}{\mm}. The load at the upstream side is put on receivers prepared in the inner vessel of the CS when it is inserted and positioned in the CS.
The central vacuum flange attached to the mechanical flange is made of aluminum to suppress activation, and it is equipped with a beam window and two feedthroughs for cooling pipes.
It also serves as the interface for a target system, which is designed to be exchangeable independently of the CS inner shield, with a precise position reference. 
The whole system is also designed to be exchangeable with semi-remote handling after Phase-I, at which the components will be highly radio-activated. 
The vacuum sealing between the CS flange and the mechanical flange of the CS inner shield system is made remotely using a \SI{1200}{\mm} bore pillow seal, which was developed for this purpose \cite{pillow-seal}.


%The shield is supported by two beams coupled to a mechanical flange system, where several vacuum flanges are mounted.
%Designing this system (downstream flange ASSY) was also completed in FY2024.
To check the mechanical strength of the system, we performed a structural analysis, and the original design (shown in \figref{fig:system}) turned out to be insufficient against the horizontal load from the large-bore pillow seal and the seismic load.
 The thickness of the mechanical flange is increased from \SI{70}{\mm} to \SI{120}{\mm} and ribs are added.
With this design, the deformation around the vacuum seal is less  \SI{<50}{\micro\metre}, which is less than the specification of U-TIGHT seal of \SI{100}{\micro\metre}.
The maximum stress remains below the allowable tensile stress\footnote{JIS B 8265 Construction of pressure vessel --- General principles} 
(\SI{129}{\newton\per\square\mm}) as shown in Fig.~\ref{fig:CSInnerShield_StructuralAnalysis}(a).
In the vertical and axial directions, the maximum stress in the welded section is within the allowable tensile stress for welds (\SI{58.1}{\newton\per\square\mm}).
In the horizontal direction, the maximum stress in the welded section does not exceed the allowable bending stress (\SI{87.1}{\newton\per\square\mm}) as shown in Fig.~\ref{fig:CSInnerShield_StructuralAnalysis}(b).
\begin{figure}[tbp]
\centering
\includegraphics[width=1\linewidth]{fig/CSInnerShield_StructuralAnalysis.pdf}
\caption{The von Mises stress against the horizontal seismic load (0.4~G). (a) Maximum stress of \SI{117.5}{\newton\per\square\mm} is applied to a positioning pin. (b) Maximum stress within the welded section is \SI{78.9}{\newton\per\square\mm}, which is achieved by adding the rib structures.}
\label{fig:CSInnerShield_StructuralAnalysis}
\end{figure}


\subsubsection{Installation and extraction method}
We designed the method for installation and extraction to minimize the exposure of workers to the residual dose.
The residual dose after the Phase-I operation was calculated and evaluated with a PHITS \cite{phits} simulation.
The CS inner shield system is inserted into and extracted from CS using a crane and a dedicated cantilever lifting jig.
The clearance between the CS bore and the diameter of the CS inner shield is \SI{\pm 10}{\mm}. 
The position and posture of the shield during insertion/extraction are constrained at four points by guides precisely aligned in advance. The four guide points are also equipped with active cameras to remotely monitor the position and posture.
A  \SI{100}{\mm} thick iron shielding wall is placed between the mechanical flange of the CS inner shield system and the lifting jig so that workers can approach the flange. The connection/disconnection at the feedthroughs and closing/opening the aluminum flange will be done over the shielding wall.
The alignment of the guide and the measurement of the final positioning will be done using a 3D coordinate measuring machine. Because several reference markers embedded in the beam room were measured in advance and the coordinate system of the beam line was calibrated in the machine, the survey in situ can be done quickly in a limited space.


\section{Radiation damage to the coils}
\begin{figure}[tbp]
  \centering
  \subfloat[Heat load]{%
    \includegraphics[width=0.5\textwidth]{fig/CSHeatLoad.pdf}
  }
   \subfloat[Fluence of neutrons (\SI{>0.1}{\MeV}) for 150 days]{%
    \includegraphics[width=0.5\textwidth]{fig/CSNeutronFluence.pdf}
  }
  \caption{Impact on the superconducting coils for an exposure to \SI{3.2}{\kilo\watt} beam with a 490-mm long graphite target and \SI{4.4}{\tesla} CS field. The horizontal axes show the different coil regions, from the upstream side of CS (with respect to the secondary beam) to the bridge solenoid. The CS coils are subdivided into 12 sections in the azimuthal direction.}
  \label{fig:CSImpact}
\end{figure}
The impact of the radiation on the superconducting coils was studied with PHITS (version 3.35.1).
\Figref{fig:CSImpact}(a) shows the energy deposit density in the coils for different regions of the coils with the CS inner shield.
The peaks correspond to the 8 and 9 o'clock positions, where the OFC modules locate.
The total heat load is calculated by summing the deposit density with weights from the design values of the coils since the implementation of coil geometry in the PHITS simulation is not perfect. 
The total heat load is  \SI{\sim 13.6}{\watt} (not including a safety factor).
The neutron fluence is also evaluated as shown in \figref{fig:CSImpact}(b). Here neutrons \SI{>0.1}{\MeV} are counted because they are known to damage the coils \cite{pcs-irradiation}.  The fluence in a 150-day beam exposure is $\Phi_n \sim 1\times10^{16}~\unit{\per\square\cm}$ at the maximum position.


\begin{figure}[tbp]
  \centering
  \subfloat[Heat load]{%
    \includegraphics[width=0.5\textwidth]{fig/CSHeatLoadComp.pdf}
  }
   \subfloat[Fluence of neutrons (\SI{>0.1}{\MeV}) for 150 days]{%
    \includegraphics[width=0.5\textwidth]{fig/CSNeutronFluenceComp.pdf}
  }
  \caption{Comparison of the impact on the superconducting coils with and without the CS inner shield. }
  \label{fig:CSImpactComp}
\end{figure}
As a comparison, we also calculated these values without the CS inner shield to see the effects.
\Figref{fig:CSImpactComp} shows the results.
The effect of the CS inner shield is the highest at the highest heat load position, where the heat load is suppressed by a factor \num{\sim 20}. The total heat load is decreased by a factor 11.
On the other hand, the neutron fluence is not drastically changed.


\section{Summary}
To make the highest intensity muon beam using the J-PARC high-power proton beam and a novel pion capture system,
a dedicated radiation shielding system has been designed and developed.
A design that meets the requirements for accepting \SI{3.2}{\kW} beam has been completed.
The production and construction of the system began this year.
As of writing, four SS304 modules are being manufactured.
The system is extendable to accept higher beam power by upgrading the shield material with heavier material with keeping the concept of the design.


\appendix

\section{Timeline}

\begin{description}[itemsep=3pt, parsep=0pt]
  \item[2024 Jul.] Bidding for the first four SS304 modules, upstream and downstream endplates (Japan).
  \item[2024 Oct.] Approval of the change in the shield shape.
  \item[2025 Mar.] Delivery of the first four SS304 modules, upstream and downstream endplates (Japan).
  \item[2025 Nov.] Fix the design and specifications for the rest four SS304 modules (China).
\end{description}

\section{Drawings} \label{sec:drawings}
Figures~\ref{fig:SS304ASSY}--\ref{fig:AluminumFlangePart} show the drawings.
They are drawn using the method of the third-angle projection.
Be careful for the treatment of the following drawings.

\begin{figure}[p]
\centering
\includegraphics[angle=90,width=1\linewidth]{fig/Fig2-1ShieldMain_SUS_v4.pdf}
\caption{Assembly drawing of SS304 module.}
\label{fig:SS304ASSY}
\end{figure}

\begin{figure}[p]
\centering
\includegraphics[angle=90,width=1\linewidth]{fig/SS304PartDrawing.pdf}
\caption{Part drawing of SS304 module main body.}
\label{fig:SS304Part}
\end{figure}


\begin{figure}[p]
\centering
\includegraphics[angle=90,width=1\linewidth]{fig/Fig1-1ShieldMain_Cu_v2.pdf}
\caption{Assembly drawing of OFC module.}
\label{fig:OFCASSY}
\end{figure}


\begin{figure}[p]
\centering
\includegraphics[angle=90,width=1\linewidth]{fig/Fig1-2ShieldMain_Cu.pdf}
\caption{Part drawing of OFC module left body.}
\label{fig:OFCPart}
\end{figure}

\begin{figure}[p]
\centering
\includegraphics[angle=90,width=1\linewidth]{fig/Fig3-3Upstream Endplate.pdf}
\caption{Part drawing of the upstream endplate.}
\label{fig:UpstreamEndplatePart}
\end{figure}

\begin{figure}[p]
\centering
\includegraphics[angle=90,width=1\linewidth]{fig/Fig3-4Downstream Endplate.pdf}
\caption{Part drawing of the downstream endplate.}
\label{fig:DownstreamEndplatePart}
\end{figure}


\begin{figure}[p]
\centering
\includegraphics[angle=90,width=1\linewidth]{fig/SupportASSY.pdf}
\caption{Assembly drawing of mechanical support.}
\label{fig:SupportASSY}
\end{figure}

\begin{figure}[p]
\centering
\includegraphics[angle=90,width=1\linewidth]{fig/MechanicalFlange.pdf}
\caption{Part drawing of the mechanical flange.}
\label{fig:MechanicalFlangePart}
\end{figure}

\begin{figure}[p]
\centering
\includegraphics[angle=90,width=1\linewidth]{fig/SupportParts.pdf}
\caption{Part drawing of the support beams and the support flange.}
\label{fig:SupportParts}
\end{figure}


\begin{figure}[p]
\centering
\includegraphics[angle=90,width=1\linewidth]{fig/AluminumFlangeASSY.pdf}
\caption{Assembly drawing of vacuum flange system.}
\label{fig:AluminumFlangeASSY}
\end{figure}

\begin{figure}[p]
\centering
\includegraphics[angle=90,width=1\linewidth]{fig/AluminumFlangePart.pdf}
\caption{Part drawing of the aluminum flange.}
\label{fig:AluminumFlangePart}
\end{figure}


%You can use this file as a template to prepare your manuscript for JPS Conference Proceedings\cite{cp,jpsj,ptep,instructions,format}.
%
%Copy \verb|jps-cp.cls| and \verb|cite.sty| onto an arbitrary directory under the texmf tree, for example, \verb|$texmf/tex/latex/jpsj|. If you have already obtained \verb|cite.sty|, you do not need to copy it.
%
%Many useful commands for equations are available because \verb|jps-cp.cls| automatically loads the \verb|amsmath| package. Please refer to reference books on \LaTeX\ for details on the \verb|amsmath| package.
%
%The \verb|twocolumn| option is not available in this class file.
%
%\section{Another Section}
%\subsection{Subsection}
%\subsubsection{Subsubsection}
%
%
%\begin{table}[tbh]
%\caption{Captions to tables and figures should be sentences.}
%\label{t1}
%\begin{tabular}{ll}
%\hline
%AAA & BBB \\
%CCC & DDD \\
%\hline
%\end{tabular}
%\end{table}
%
%\subsubsection{Equation numbers}
%
%The \verb|seceq| option resets the equation numbers at the start of each section.
%
%
%
%Label figures, tables, and equations appropriately using the \verb|\label| command, and use the \verb|\ref| command to cite them in the text as ``\verb|as shown in Fig. \ref{f1}|". This automatically labels the numbers in numerical order.
%
%The \verb|minipage| environment can be used to place figures horizontally.
%
%\begin{equation}
%E = mc^{2}
%\label{e1}
%\end{equation}

% \appendix
% \section{}

% Use the \verb|\appendix| command if you need an appendix(es). The \verb|\section| command should follow even though there is no title for the appendix (see above in the source of this file).

%\section*{Acknowledgment}
%This work was supported by JSPS KAKENHI Grant Number JP22K21350.

\begin{thebibliography}{9}
\bibitem{pcs} M. Yoshida, et al., IEEE Trans. Appl. Supercond. \textbf{25}, 4500904 (2015).
\bibitem{inner-shield} Y. Yang, et al., IEEE Trans. Appl. Supercond. \textbf{28}, 4001405 (2018).
\bibitem{comet} R. Abramishvili, et al. (COMET collaboration), Prog. Theor. Exp. Phys. \textbf{2020}, 033C01 (2020).
\bibitem{pillow-seal} R. Kurasaki, et al., PASJ Conf. Proc. \textbf{2022}, 521 WEP007 (2022).
\bibitem{phits} T. Sato, et al., J. Nucl. Sci. Technol. \textbf{61}, 127 (2024).
\bibitem{pcs-irradiation}M. Yoshida, et al., in Proc. AIP Conf. \textbf{1435}, 167--173 (2011).
% \bibitem{cp} The abbreviation for JPS Conference Proceedings should be ``JPS Conf. Proc." in the reference list.
% \bibitem{jpsj} The abbreviation for the Journal of the Physical Society of Japan should be ``J. Phys. Soc. Jpn." in the reference list.
% \bibitem{ptep} The abbreviation for the Progress of Theoretical and Experimental Physics should be ``Prog. Theor. Exp. Phys." in the reference list.
% \bibitem{instructions} More abbreviations of journal titles are listed in ``Instructions for Preparation of Manuscript", which is available at our Web site (http://jpsj.jps.or.jp).
% \bibitem{format} F. Author, S. Author, and T. Author, Abbreviated journal title \textbf{volume in bold face}, initial page or article number (year of publication).
\end{thebibliography}

\end{document}

